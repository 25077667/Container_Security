\documentclass[12pt,a4paper,oneside,draft]{IEEEconf}
\usepackage[useregional]{datetime2}
\usepackage[pdf]{graphviz}
\usepackage{hyperref}
\usepackage[backend=biber,style=numeric,sorting=none]{biblatex}
\usepackage{pstool}
\usepackage[final]{listings}
\usepackage{color}
\usepackage{amsmath}
\usepackage{xparse}

\definecolor{mygreen}{rgb}{0,0.6,0}
\definecolor{mygray}{rgb}{0.5,0.5,0.5}
\definecolor{mymauve}{rgb}{0.58,0,0.82}
\addbibresource{mergedBib.bib}

% Link conf.
\hypersetup{
    citecolor=red,
    colorlinks=true,
    linkcolor=blue,
    filecolor=magenta,      
    urlcolor=cyan,
    pdfpagemode=FullScreen,
}
\urlstyle{same}

% Code conf.
\lstset{ 
  backgroundcolor=\color{white},   % choose the background color; you must add \usepackage{color} or \usepackage{xcolor}; should come as last argument
  basicstyle=\ttfamily\footnotesize,% the size of the fonts that are used for the code
  breakatwhitespace=false,         % sets if automatic breaks should only happen at whitespace
  breaklines=true,                 % sets automatic line breaking
  captionpos=b,                    % sets the caption-position to bottom
  commentstyle=\color{mygreen},    % comment style
  deletekeywords={...},            % if you want to delete keywords from the given language
  escapeinside={\%*}{*)},          % if you want to add LaTeX within your code
  extendedchars=true,              % lets you use non-ASCII characters; for 8-bits encodings only, does not work with UTF-8
  frame=single,	                   % adds a frame around the code
  keepspaces=true,                 % keeps spaces in text, useful for keeping indentation of code (possibly needs columns=flexible)
  keywordstyle=\color{red},       % keyword style
  morekeywords={*,...},            % if you want to add more keywords to the set
  numbers=left,                    % where to put the line-numbers; possible values are (none, left, right)
  numbersep=5pt,                   % how far the line-numbers are from the code
  numberstyle=\tiny\color{mygray}, % the style that is used for the line-numbers
  rulecolor=\color{black},         % if not set, the frame-color may be changed on line-breaks within not-black text (e.g. comments (green here))
  showspaces=false,                % show spaces everywhere adding particular underscores; it overrides 'showstringspaces'
  showstringspaces=false,          % underline spaces within strings only
  showtabs=false,                  % show tabs within strings adding particular underscores
  stepnumber=1,                    % the step between two line-numbers. If it's 1, each line will be numbered
  stringstyle=\color{mymauve},     % string literal style
  tabsize=4,	                     % sets default tabsize to ˋ spaces
  title=\lstname                   % show the filename of files included with \lstinputlisting; also try caption instead of title
}

\font\mytitle=cmr12 at 30pt

\title{{\mytitle Container Security}}
\author{Chih-Hsuan Yang\\

National Sun-Yet-San University, Taiwan \\
Bachelor's degree graduation project \\

Advisor: Chun-I Fan
}

\date{\today}



\begin{document}

% Cover page
\maketitle

%\newpage
%\tableofcontents
%\newpage

\section{Abstract}
%This research focuses on the container’s modern cybersecurity issue. 
Recently, many companies use containers to run their microservices, since containers could
make their hardware resources be used efficiently. For example, GCP(Google Cloud Platform),
AWS(Amazon Web Services), and Microsoft Azure are using this technique to separate subscribers'
resources and services. However, if the hacker attacks the kernel or gets privilege
escalation of containers, then such attacks would influence the host or the other containers.
Therefore, this research would analyze some kernel vulnerabilities, which could inspire in
the container and influence the host or the other containers. This project would aim and patch
a kernel vulnerability. Hope this research would make the technique of container more secure.

\section{Motivation}
The Container is a virtualization technique to package applications and dependencies to run in
an isolated environment. Containers are faster to start-up, lighter in memory/storage usage
at run time and easier to build up than virtual machines. Because the container shares the
kernel with the host OS and other containers.\\
I often used to run a docker container to host my services. For example: my homework,
servers and some services in Information security club at NSYSU.
But there are some vulnerabilities about container technique. Like "Dirty CoW\cite{Dirty_CoW}"
and "Escape vulnerabilities".\\
"Dirty CoW is a vulnerability in the Linux kernel. It is a local privilege escalation bug
that exploits a race condition in the implementation of the copy-on-write mechanism in the
kernel's memory-management subsystem"\cite{Dirty_CoW_wiki}. It founded by Phil Oester. I
was 16, the first year I had touched the docker container. I tried to use the Dirty CoW
vulnerability to take the root privilege of my Android phone.\\
Escape vulnerability is a subcategory of sandbox security. At first, security researchers often
need sandbox to help they analyze malware, which prevent the malware influence researcher's
host OS. Nowadays, the sandbox not only be used in analyzing, but also used to execute a
normal application for an isolated environment. But if the application could modify the
outside resources without the kernel permission. That loses the purpose of isolation. That
might cause the information leaked or the kernel be hacked.\\
Hence, there is a big problem about: "How to make sure my services isolated and secure?" The
author is the leader of Information security club. He should maintain all the services working
perfectly. Moreover we are information security club. Therefore, the security and performance
issue is the top-priority requirement.\\


\section{Papers Review}
\subsection{Study of the Dirty Copy On Write\cite{Study_Dirty_Cow}}
In this paper show the race condition, and the mechanism of "copy on write".
"Copy on write" is "A resource-management technique used in computer programming to
efficiently implement a "duplicate" or "copy" operation on modifiable resources."
\cite{CoW_wiki} We often use the CoW while fork() or mmap().

\subsubsection{mmap}
This is a system call of mapping files or devices in to memory, which creates a
new mapping in the virtual address space of the calling process. Some libiraries
are also mapped into the virtual address space to handle the function call. 

\paragraph{Why use mmap}
\paragraph{How to use mmap}

\subsubsection{Copy on write}
// FIXME: Many callers request same modifiable resources, and the kernel could use this
technique to enhance the performance of these callers.

\subsubsection{Race condition}
// FIXME: Processes or threads are racing the same modifiable resources.

\subsubsection{\hypertarget{dirty cow}{Dirty CoW} demo code\cite{Dirty_CoW}}
Let's analyze the proof of concept(PoC) of dirty CoW.(Oester, 2016)\\
The key of inspiring this vulnerability is the mmaped memory space, which is mapped with
the PROT\_READ flag. The PROT\_READ flag declares the page is read only.
\lstinputlisting[language=C, linerange={87-89, 101-101}, firstnumber=87]{dirtyc0w.c}

It creates 2 threads, which would have a race condition of the mmaped memory space,
\hyperlink{madvise}{madviseThread} and \hyperlink{procself}{procselfmemThread}.

\hypertarget{threads_main}{threads in main}
\lstinputlisting[language=C, linerange={106-107}, firstnumber=106]{dirtyc0w.c}

In one thread, call a system call "madvise", would make the user thread gain the root
privilege to operate the protected page temporary. And the flag MADV\_DONTNEED would
tell the kernel: "Do not Expected access it in the near future.\cite{Madvise}" Moreover,
this flag might not lead to immediate freeing of pages in the range.The kernel is free
to delay free the pages until an appropriate moment.\cite{Madvise}

\hypertarget{madvise}{madviseThread}
\lstinputlisting[language=C, linerange={33-39,45-48}, firstnumber=33]{dirtyc0w.c}

In another thread, open its memory resource file. This file is a special file, which allow
the process reads its memory by itself.\\
Than, we move the printer of file descriptor of the memory resource file to the mmaped
space. And try to write it. But the mmaped space is a read only space. We expected the
kernel would create a copy of the this space and write the copy\cite{root_exploit}.
\hypertarget{procself}{procselfmemThread}
\lstinputlisting[language=C, linerange={50-53,61-63,67-71}, firstnumber=50]{dirtyc0w.c}

But there is a problem! There is an another thread is racing this page with root privilege.
If the schedular context switches the madviseThread to procselfmemThread, while the
adviseThread is calling the "madvise" system call. It would cause the procselfmemThread
gain the root privilege from madviseThread to control the mmaped file.

\subsection{Container Security: Issues, Challenges, and the Road Ahead\cite{Road_Ahead}}
This paper has derived 4 generalized container security issues: (\RN{1}) protecting a
container from applications inside it, (\RN{2}) inter-container protection, (\RN{3})
protecting the host from containers, and (\RN{4}) protecting containers from a malicious
or semi-honest host.\cite{Road_Ahead}

The \hyperlink{dirty cow}{Dirty CoW} vulnerability is a exploit from kernel. But the benefit
of container and host OS are share the same kernel. This vulnerability can be used in
container to attack the kernel, and gives this application root privilege, changes this
containers as a privileged container or supervises the other containers. Therefore, we should
protect the host form the container(which belongs to type (\RN{3}) threat in this paper).

\subsubsection{Virtual machine and container}
// FIXME: Draw the architecture of VM and container.

\subsubsection{Linux kernel features}
// FIXME: Introduce these features for isolating processes in Linux.

\paragraph{namespaces}\mbox{}\\
// FIXME: Namespaces perform the job of isolation and virtualization of system resources
for a collection of processes.\cite{Road_Ahead}

\paragraph{cgroups}\mbox{}\\
// FIXME: Limits, accounts for, and isolates the resource usage of a collection of processes.
\cite{cgroup_wiki}

\paragraph{capabilities}\mbox{}\\
// FIXME: Divide the privileges traditionally associated with superuser into distinct
units.

\paragraph{seccomp}\mbox{}\\
// FIXME: Only some specified process could call some specified system calls.


\section{Methods}
\subsection{Study CVEs about the Linux kernel}
This project will study several kernel vulnerabilities were discover in packet socket: CVE-2016-8655
\cite{CVE-2016-8655}, CVE-2017-7308\cite{CVE-2017-7308}, and CVE-2020-14386\cite{CVE-2020-14386}.

And study some kernel exploit techniques\cite{Kernel_exploitation}, because the container shares
the kernel. If I could exploit the kernel in the suffering container, it might have more chance
to influence the other containers or host.

\subsection{Study related mechanisms}
The Linux kernel is a monolithic kernel, which is over 28 million lines of codes now(2020). There
are many mechanisms to solve the real life situations. Study those CVEs' related mechanism in the
kernel, might have more chance to find new vulnerabilities.

\subsection{Implement a simple container}

\subsection{List secure issues of the simple container}

\subsection{Aim a vulnerability}
After studying the CVEs and related mechanisms. This project would find a vulnerability of privilege
escalation in the container by attacking the kernel.

\subsection{Implement the PoC}
Implement the proof of concept of the aimed vulnerability.

\subsection{Implement the patch and pull request}
Being a security researcher, we cannot just only exploit the software, but also give patches to
the maintainer. The Linux kernel is an open source project under GPL-2.0 license in GitHub.
I would pull request to the repository. If my patch could be merged into the kernel to solve the
container vulnerability.

\section{Expected Outcome}
Would research some related vulnerabilities, and implement the PoC code.
Moreover this project will generate the patch of the vulnerability(s) to protect these attack(s).

\printbibheading[heading=bibnumbered]
\printbibliography\newrefcontext

\section{Academic Advisor}
\begin{itemize}
  \item Organize to a complete structure.
  \item Extend to a formal paper, and publish.
\end{itemize}

\end{document}
