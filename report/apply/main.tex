\documentclass[12pt,a4paper,oneside,draft]{article}
\usepackage[useregional]{datetime2}
\usepackage[pdf]{graphviz}
\usepackage{hyperref}
\usepackage[backend=biber,style=numeric,sorting=none]{biblatex}
\usepackage{pstool}
\usepackage[final]{listings}
\usepackage{color}

\definecolor{mygreen}{rgb}{0,0.6,0}
\definecolor{mygray}{rgb}{0.5,0.5,0.5}
\definecolor{mymauve}{rgb}{0.58,0,0.82}
\addbibresource{mergedBib.bib}

% Link conf.
\hypersetup{
    citecolor=red,
    colorlinks=true,
    linkcolor=blue,
    filecolor=magenta,      
    urlcolor=cyan,
    pdfpagemode=FullScreen,
}
\urlstyle{same}

\lstset{ 
  backgroundcolor=\color{white},   % choose the background color; you must add \usepackage{color} or \usepackage{xcolor}; should come as last argument
  basicstyle=\footnotesize,        % the size of the fonts that are used for the code
  breakatwhitespace=false,         % sets if automatic breaks should only happen at whitespace
  breaklines=true,                 % sets automatic line breaking
  captionpos=b,                    % sets the caption-position to bottom
  commentstyle=\color{mygreen},    % comment style
  deletekeywords={...},            % if you want to delete keywords from the given language
  escapeinside={\%*}{*)},          % if you want to add LaTeX within your code
  extendedchars=true,              % lets you use non-ASCII characters; for 8-bits encodings only, does not work with UTF-8
  frame=single,	                   % adds a frame around the code
  keepspaces=true,                 % keeps spaces in text, useful for keeping indentation of code (possibly needs columns=flexible)
  keywordstyle=\color{blue},       % keyword style
  morekeywords={*,...},            % if you want to add more keywords to the set
  numbers=left,                    % where to put the line-numbers; possible values are (none, left, right)
  numbersep=5pt,                   % how far the line-numbers are from the code
  numberstyle=\tiny\color{mygray}, % the style that is used for the line-numbers
  rulecolor=\color{black},         % if not set, the frame-color may be changed on line-breaks within not-black text (e.g. comments (green here))
  showspaces=false,                % show spaces everywhere adding particular underscores; it overrides 'showstringspaces'
  showstringspaces=false,          % underline spaces within strings only
  showtabs=false,                  % show tabs within strings adding particular underscores
  stepnumber=1,                    % the step between two line-numbers. If it's 1, each line will be numbered
  stringstyle=\color{mymauve},     % string literal style
  tabsize=4,	                   % sets default tabsize to 2 spaces
  title=\lstname                   % show the filename of files included with \lstinputlisting; also try caption instead of title
}

\font\mytitle=cmr12 at 30pt

\title{{\mytitle Container Security}}
\author{Chih-Hsuan Yang\\

National Sun-Yet-San University, Taiwan \\
Bachelor's degree graduation project \\

Advisor: Chun-I Fan
}

\date{\today}



\begin{document}

% Cover page
\maketitle

\newpage
\tableofcontents
\newpage

\section{Abstract}
A research of container's modern cyber security issue.
Many companies use container to run their services.\\
// FIXME

\section{Motivation}
The Container is a virtualization technique to package applications and dependencies to run in
an isolated environment. Containers are faster to start-up, lighter in memory/storage usage
at run time and easier to build up than virtual machines. Because the container shares the
kernel with the host OS and other containers.\\
I often used to run a docker container to host my services. For example: my homework,
servers and some services in Information security club at NSYSU.
But there are some vulnerabilities about container technique. Like "Dirty CoW\cite{Dirty_CoW}"
and "Escape vulnerabilities".\\
"Dirty CoW is a vulnerability in the Linux kernel. It is a local privilege escalation bug
that exploits a race condition in the implementation of the copy-on-write mechanism in the
kernel's memory-management subsystem"\cite{Dirty_CoW_wiki}. It founded by Phil Oester. I
was 16, the first year I had touched the docker container. I tried to use the Dirty CoW
vulnerability to take the root privilege of my Android phone.\\
Escape vulnerability is a subcategory of sandbox security. At first, security researchers often
need sandbox to help they analyze malware, which prevent the malware influence researcher's
host OS. Nowadays, the sandbox not only be used in analyzing, but also used to execute a
normal application for an isolated environment. But if the application could modify the
outside resources without the kernel permission. That loses the purpose of isolation. That
might cause the information leaked or the kernel be hacked.\\
Hence, there is a big problem about: "How to make sure my services isolated and secure?" I
am the leader of Information security club. I should maintain all the services working
perfectly. Moreover we are information security club. Therefore, the security and performance
issue is the top-priority requirement.\\


\section{Papers Review}
\subsection{Study of the Dirty Copy On Write\cite{Study_Dirty_Cow}}
In this paper show the race condition, and the mechanism of "copy on write".
"Copy on write" is "A resource-management technique used in computer programming to
efficiently implement a "duplicate" or "copy" operation on modifiable resources."
\cite{CoW_wiki} We often use the CoW while fork() or mmap().

\subsubsection{mmap}

\subsubsection{race condition}

\subsubsection{Dirty CoW demo code\cite{Dirty_CoW}}
Let's analyze the Proof of concept (POC) of dirty CoW.\\
There are 2 function have race condition, \hyperlink{madvise}{madviseThread} and
\hyperlink{procself}{procselfmemThread}.
\lstinputlisting[language=C, linerange={106-107}]{dirtyc0w.c}

Create 2 thread and give them argvs.\\
\newline

\hypertarget{madvise}{madviseThread}
\lstinputlisting[language=C, linerange={33-39,45-48}]{dirtyc0w.c}

The madvise() is a system call, which gives advice about use of memory.
The MADV\_DONTNEED is a flag to tell kernel "Do not expect access it in the near
future." And the manual said: "Note that, when applied to shared mappings,
MADV\_DONTNEED might not lead to immediate freeing of the pages in the range. The
kernel is free to delay freeing the pages until an appropriate moment.\cite{Madvise}"\\

That is the problem! We ask kernel we would not access it in the near future. The kernel
would delay a moments to free it. But, at the meanwhile, our another thread asks for write
it.

\hypertarget{procself}{procselfmemThread}
\lstinputlisting[language=C, linerange={50-53,61-63,67-71}]{dirtyc0w.c}
If the scheduler context switch from \hyperlink{madvise}{madviseThread} to the
\hyperlink{procself}{procselfmemThread}, while the madvise() just was having been called.
The permission of the page of the mmaped file would not change.

Therefore we could get the permission for I/O.

\subsection{Linux Kernel OS Local Root Exploit\cite{root_exploit}}

\section{Methods}

\section{Expected Outcome}
Generate the PoC code and the solution of a container cyber attack.

\printbibheading[heading=bibnumbered]
\printbibliography[sorting=none]

\end{document}