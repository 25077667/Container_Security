\documentclass[12pt,a4paper,oneside,draft]{article}
\usepackage[useregional]{datetime2}

\usepackage{hyperref}
\usepackage{biblatex}
\addbibresource{mergedBib.bib}

% Link conf.
\hypersetup{
    citecolor=red,
    colorlinks=true,
    linkcolor=blue,
    filecolor=magenta,      
    urlcolor=cyan,
    bookmarks=true,
    pdfpagemode=FullScreen,
}
\urlstyle{same}

\font\mytitle=cmr12 at 30pt

\title{{\mytitle Container Security}}
\author{Chih-Hsuan Yang}
\date{\today}



\begin{document}

% Cover page
\maketitle
\begin{center}
    \begin{large}
        National Sun-Yet-San University \\
        Bachelor's degree graduation project \\
    \end{large}
    Advisor: Chun-I Fan
\end{center}

\newpage

\tableofcontents
\newpage

\section{Abstract}
A research of container's modern cyber security issue.
Many companies use container to run their services.\\
// FIXME

\section{Motivation}

Container is a virtualization technique to package applications and dependencies to run in
an isolated environment. Containers are faster to start up, lighter in memory/storage usage
at run time and easier to build up than virtual machines. Because the container share the
kernel with the host OS and other containers.
I often used to run a docker container to host my services. For example: my homeworks,
servers and some services in Information security club at NSYSU.
But there are some vulnerabilities about container technique. Like "Dirty CoW\cite{Dirty_CoW}"
and "Escape vulnerabilities".
Hence there is a big problem about: "How to make sure my services are isolated and secure
?" I am the leader of Information security club, I should maintain all the
services working perfectly. Moreover we are information security club. Hence the security
and performance issue are the top priority requirements.\\


\section{Papers Review}

\section{Methods}

\section{Expected Outcome}
The PoC code and the solution of a container cyber attack.

\printbibheading[heading=bibnumbered]
\printbibliography

\end{document}