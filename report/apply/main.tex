\documentclass[12pt,a4paper,oneside]{IEEEconf}
\usepackage[useregional]{datetime2}
\usepackage[pdf]{graphviz}
\usepackage{hyperref}
\usepackage[backend=biber,style=numeric,sorting=none]{biblatex}
\usepackage{pstool}
\usepackage[final]{listings}
\usepackage{color}
\usepackage{amsmath}
\usepackage{xparse}
\usepackage[T1]{fontenc}
\usepackage[utf8]{inputenc}
\usepackage{enumerate}
\graphicspath{{./images/}}

\definecolor{mygreen}{rgb}{0,0.6,0}
\definecolor{mygray}{rgb}{0.5,0.5,0.5}
\definecolor{mymauve}{rgb}{0.58,0,0.82}
\addbibresource{mergedBib.bib}

% Link conf.
\hypersetup{
    citecolor=blue,
    colorlinks=true,
    linkcolor=blue,
    filecolor=magenta,      
    urlcolor=cyan,
    pdfpagemode=FullScreen,
}
\urlstyle{same}

% Code conf.
\lstset{ 
  backgroundcolor=\color{white},   % choose the background color; you must add \usepackage{color} or \usepackage{xcolor}; should come as last argument
  basicstyle=\ttfamily\footnotesize,% the size of the fonts that are used for the code
  breakatwhitespace=false,         % sets if automatic breaks should only happen at whitespace
  breaklines=true,                 % sets automatic line breaking
  captionpos=b,                    % sets the caption-position to bottom
  commentstyle=\color{mygreen},    % comment style
  deletekeywords={...},            % if you want to delete keywords from the given language
  escapeinside={\%*}{*)},          % if you want to add LaTeX within your code
  extendedchars=true,              % lets you use non-ASCII characters; for 8-bits encodings only, does not work with UTF-8
  frame=single,	                   % adds a frame around the code
  keepspaces=true,                 % keeps spaces in text, useful for keeping indentation of code (possibly needs columns=flexible)
  keywordstyle=\color{red},        % keyword style
  morekeywords={*,...},            % if you want to add more keywords to the set
  numbers=left,                    % where to put the line-numbers; possible values are (none, left, right)
  numbersep=5pt,                   % how far the line-numbers are from the code
  numberstyle=\tiny\color{mygray}, % the style that is used for the line-numbers
  rulecolor=\color{black},         % if not set, the frame-color may be changed on line-breaks within not-black text (e.g. comments (green here))
  showspaces=false,                % show spaces everywhere adding particular underscores; it overrides 'showstringspaces'
  showstringspaces=false,          % underline spaces within strings only
  showtabs=false,                  % show tabs within strings adding particular underscores
  stepnumber=1,                    % the step between two line-numbers. If it's 1, each line will be numbered
  stringstyle=\color{mymauve},     % string literal style
  tabsize=4,	                     % sets default tabsize to ˋ spaces
  title=\lstname,                  % show the filename of files included with \lstinputlisting; also try caption instead of title
  extendedchars=true
}

\font\mytitle=cmr12 at 30pt

\title{{\mytitle Container Security}}
\author{Chih-Hsuan Yang\\
National Sun-Yet-San University, Taiwan \\
Bachelor's degree graduation project \\
Advisor: Chun-I Fan
}

\date{\today}



\begin{document}

% Cover page
\maketitle

%\newpage
%\tableofcontents
%\newpage

% ================ Abstract ================

\section{Abstract}
Recently, many companies use containers to run their microservices, since containers could
make their hardware resources be used efficiently. For example, GCP(Google Cloud Platform),
AWS(Amazon Web Services), and Microsoft Azure are using this technique to separate subscribers'
resources and services. However, if the hacker attacks the kernel or gets privilege
escalation of containers, then such attacks would influence the host or the other containers.
Therefore, this research would analyze, implement, and protect the container escalation.
The container escalation could inspire in a container and influence the host or the
other containers. This project would use the medical system with FHIR(Fast Healthcare
Interoperability Resources) to simulate the real world threat and purpose a secure solution
to protect patient's privacy.

% ================ Motivation ================

\section{Motivation}
The Container is a virtualization technique to package applications and dependencies to run in
an isolated environment. Containers are faster to start-up, lighter in memory/storage usage
at run time and easier to deploy than virtual machines. Because the container shares the
kernel with the host OS and other containers, and deploys by a configure file.\\
First, we often used to run a docker container to host our services. For example: assignments,
servers and some services in Information security club at NSYSU(National Sun Yat-sen University).
But there are some threats about container technique. Like "Dirty CoW\cite{Dirty_CoW}"
and "Escape vulnerabilities".\\
"Dirty CoW is a vulnerability in the Linux kernel. It is a local privilege escalation bug
that exploits a race condition in the implementation of the copy-on-write mechanism in the
kernel's memory-management subsystem"\cite{Dirty_CoW_wiki}. It founded by Phil Oester. We
were 16, the first year we had touched the docker container. We tried to use the Dirty CoW
vulnerability to take the root privilege of my Android phone.\\
Escape vulnerability is a subcategory of sandbox security. At first, security researchers often
need sandbox to help they analyze malware, which prevent the malware influence researcher's
host OS. Nowadays, the sandbox not only be used in analyzing, but also used to execute a
normal application for an isolated environment. However if the application could modify the
outside resources without the kernel permission. That loses the purpose of isolation. That
might cause the information leaked or the kernel be hacked.\\
Hence, there is a big problem about: "How to make sure my services isolated and secure?" The
author of this paper is the leader of Information security club. He should maintain all
the services working perfectly. Moreover we are information security club. Therefore,
the security and performance issue is the top-priority requirement.\\
Second, in order to present the container security, we would take the medical system for
example.
The medical system is the most famous part internationally. Including face the COVID-19
in Taiwan, we not have the local COVID-19 case in more than 250 days.\cite{COVID19_CNN}
However, the medical profession needs to renew the exchanging EHR(electronic health records)
system these years. In order to protect the privacy of patients, and producing the high
performance system to exchange the EHR, we need a easy deployment, effective runtime, and
secure system. We have to do this research in this project.

% ================ Paper review ================

\section{Papers Review}
\subsection{Introduction}
In this section will introduce why we choose these papers, some what did the paper say, and
some related mechanisms about that issue.
In this project will focus on (\RN{1}) \hyperlink{security}{security}, (\RN{2}) \
\hyperlink{easy_management}{easy management}, \
and (\RN{3}) \hyperlink{heigh_performance}{high performance}.

\hypertarget{security}{\subsection{Security}}
\subsubsection{Study of the Dirty Copy On Write}
In this paper\cite{Study_Dirty_Cow} show the race condition, and the mechanism of "copy on write".
"Copy on write" is " a resource-management technique used in computer programming to
efficiently implement a "duplicate" or "copy" operation on modifiable resources."
\cite{CoW_wiki} It often be inspire when fork() or mmap().

\paragraph{mmap}
This is a system call of mapping files or devices in to memory, which creates a
new mapping in the virtual address space of the caller process. Such that
the process could operate the instance of file in memory directly.
And some libraries are also mapped into the virtual address space to share and handle
the function call. Therefore, the processes could take the same view of libraries in
it's memory space.

\paragraph{Copy on write}
This mechanism purposes of a resource being duplicated but not modified, it is not
necessary to create a new resource. Therefore the kernel can make callers share
the same memory resources. The mmap is a system call could inspire this mechanism
in above paragraph. Some processes request same memory resources, and the kernel
supplies the same memory page to callers.

\paragraph{Race condition}
That is processes or threads are racing the same modifiable resources. For example:
There is a page(4KB as default) of memory initialized as 0. And there are 2 threads
or processes share that page. Consider one of the tasks is assigning the page full of
character 'A'. However, the schedular context switches to the other task assigns that
page full of 'B'. And the schedular context switches again to the first task.
There is a problem now. What is the page for the first task looked like? Obviously,
it does not meet the expectation for the first task.

\subsubsection{Dirty CoW demo code}
Let's analyze the proof of concept(PoC) of dirty CoW.(Oester, 2016)\cite{Dirty_CoW}
The key of inspiring this vulnerability is the mmaped memory space, which is mapped with
the PROT\_READ flag. The PROT\_READ flag declares the page is read only.
\lstinputlisting[language=C, linerange={87-89, 101-101}, firstnumber=87]{src/dirtyc0w.c}

It creates 2 threads, which would have a race condition of the mmaped memory space,
\hyperlink{madvise}{madviseThread} and \hyperlink{procself}{procselfmemThread}.

\hypertarget{threads_main}{threads in main}
\lstinputlisting[language=C, linerange={106-107}, firstnumber=106]{src/dirtyc0w.c}

In one thread, call a system call "madvise", would make the user thread gain the root
privilege to operate the protected page temporary. And the flag MADV\_DONTNEED would
tell the kernel: "Do not Expected access it in the near future.\cite{Madvise}" Moreover,
this flag might not lead to immediate freeing of pages in the range. The kernel is free
to delay free the pages until an appropriate moment.\cite{Madvise}

\hypertarget{madvise}{madviseThread}
\lstinputlisting[language=C, linerange={33-39,45-48}, firstnumber=33]{src/dirtyc0w.c}

In another thread, open its memory resource file. This file is a special file, which allow
the process reads its memory by itself.\\
Than, we move the printer of file descriptor of the memory resource file to the mmaped
space. And try to write it. But the mmaped space is a read only space. We expected the
kernel would create a copy of the this space and write the copy\cite{root_exploit}.
\hypertarget{procself}{procselfmemThread}
\lstinputlisting[language=C, linerange={50-53,61-63,67-71}, firstnumber=50]{src/dirtyc0w.c}

But there is a problem! There is an another thread is racing this page with root privilege.
If the schedular context switches the madviseThread to procselfmemThread, while the
adviseThread is calling the "madvise" system call. It would cause the procselfmemThread
gain the root privilege from madviseThread to control the mmaped file.

\subsubsection{Container Security: Issues, Challenges, and the Road Ahead}
This paper\cite{Road_Ahead} has derived 4 generalized container security issues:
(\RN{1}) protecting a container from applications inside it, (\RN{2}) inter-container
protection, (\RN{3}) protecting the host from containers, and (\RN{4}) protecting containers
from a malicious or semi-honest host.\cite{Road_Ahead}

The Dirty CoW vulnerability is a exploit from kernel. But the benefit
of container and host OS are share the same kernel. This vulnerability can be used in
container to attack the kernel, and gives this application root privilege, changes this
containers as a privileged container or supervises the other containers. Therefore, we should
protect the host form the container(which belongs to type (\RN{3}) threat in this paper).

\paragraph{Virtual machine and container}
// FIXME: Draw the architecture of VM and container.

\paragraph{Linux kernel features}
// FIXME: Introduce these features for isolating processes in Linux.

\subparagraph{namespaces}\mbox{}\\
// FIXME: Namespaces perform the job of isolation and virtualization of system resources
for a collection of processes.\cite{Road_Ahead}

\subparagraph{cgroups}\mbox{}\\
// FIXME: Limits, accounts for, and isolates the resource usage of a collection of processes.
\cite{cgroup_wiki}

\subparagraph{capabilities}\mbox{}\\
// FIXME: Divide the privileges traditionally associated with superuser into distinct
units.

\subparagraph{seccomp}\mbox{}\\
// FIXME: Only some specified process could call some specified system calls.

\subsubsection{Kernel fuzzing}

\hypertarget{easy_management}{\subsection{Easy management}}
\subsubsection{A paper}

\hypertarget{heigh_performance}{\subsection{High performance}}
\subsubsection{PINE: Optimizing Performance Isolation in Container Environments}
This paper\cite{Optimizing} introduce a high throughput and low latency module


% ================ Methods ================

\section{Methods}
This project would use the MapReduce process.
\digraph{methFlow}{
  rankdir=LR;
  Study -> Security -> Merge;
  Study -> Implement -> Merge;
}
// FIXME: the width of digraph


\subsection{Security}
\subsubsection{Study CVEs and related mechanisms}
The Linux kernel is a monolithic kernel, which is over 28 million lines of codes now(2020). There
are many mechanisms to solve the real life situations. Study those CVEs' related mechanism in the
kernel, might have more chance to find new vulnerabilities.

This project will study several container vulnerabilities for example: CVE-2016-8655
\cite{CVE-2016-8655}, CVE-2016-9962\cite{CVE-2016-9962}, and CVE-2020-14386\cite{CVE-2020-14386}.

And study some kernel exploit techniques\cite{Kernel_exploitation}, because the container shares
the kernel. If I could exploit the kernel in the suffering container, it might have more chance
to influence the other containers or host.

\subsubsection{Implement a simple container}
The Linux kernel supply some system calls to clone a process(also in thread) in their own namespace
and group. We could implement a simple container by ourself, so that we can make a list of
vulnerabilities may happen.
% \begin{lstlisting}
\UseRawInputEncoding\lstinputlisting[language={}]{src/lc_out.txt}
% \end{lstlisting}

\subsubsection{List secure details of the simple container}
In my rough opinion, there are 5 types of container security risks, (\RN{1}) Host OS risks, (\RN{2})
Orchestration system risks, (\RN{3}) Container runtime risks, (\RN{4}) Registry risks, (\RN{5})
Images risks. In this stage we should research the details of those risk, and purpose some solutions.

\paragraph{Host OS risks}
\begin{itemize}
  \item Improper user permission
  \item Kernel vulnerabilities
\end{itemize}

\paragraph{Orchestration system risks}
\begin{itemize}
  \item Unbounded domain access
  \item Weak credentials
  \item Mismanaged inter-container network traffic
  \item Mixed of workload sensitivity levels
\end{itemize}

\paragraph{Container runtime risks}
\begin{itemize}
  \item Runtime software vulnerabilities
  \item Unbounded network access from containers
  \item Insecure container runtime configurations
\end{itemize}

\paragraph{Registry risks}
\begin{itemize}
  \item Insecure connections to registries
  \item Old images in registries
\end{itemize}

\paragraph{Images risks}
\begin{itemize}
  \item Image vulnerabilities
  \item Embedded malware or secrets
\end{itemize}

\subsubsection{Aim a vulnerability and implement the PoC}
After listing the risks. This project would find a vulnerability of privilege escalation in
the container and affect with other containers.

\subsubsection{Implement the patch and pull request}
Being a security researcher, we cannot just only exploit the software, but also give patches to
the maintainer. The Linux kernel is an open source project under GPL-2.0 license in GitHub.
I would pull request to the repository. If my patch could be merged into the kernel to solve the
container vulnerability.

\subsection{Merge}

% ================ Expected Outcome ================

\section{Expected Outcome}
Would research some related vulnerabilities, and implement the PoC code.
Moreover this project will generate the patch of the vulnerability(s) to protect these attack(s).

\printbibheading[heading=bibnumbered]
\printbibliography\newrefcontext

\section{Academic Advisor}
\begin{itemize}
  \item Organize to a complete structure.
  \item Extend to a formal paper, and publish.
\end{itemize}

\end{document}
