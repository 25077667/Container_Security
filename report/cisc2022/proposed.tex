\section{Proposed Scheme}
It is the programmer's responsibility to write complete unit and integration tests.
We extend the definition Test-Driven Development (TDD), which is not only red,
green, and refactor, but also "Test Do what's Designed".

\subsection{Workflow}
In short, our proposal is generating a perfectly fittable mask layer which
is coupled with the healthcare data exchange system in build time.

\begin{figure}[h]
    \centering
    \smartdiagramset{
        module minimum width=3cm,
        text width=2.5cm,
    }
    \smartdiagram[circular diagram:clockwise]{
        Scan base, Sign, Image checking, Start policy, Runtime enforcement
    }
    \blodcaption{Contiguous Integration and Contiguous Deployment}
    \label{workflow}
\end{figure}

We proposed a CI/CD workflow to guarantee the runtime enforcement of policies
in figure \ref{workflow}. Each block of the workflow will be described
in the following subsubsection.

Because of the CI/CD workflow, we can rolling update all the features or
fixing vulnerabilities, such that, the software would be released secure
eventually. Linus Torvalds said\footnote{\url{https://www.youtube.com/watch?v=5CIL54-KKz0}}
: "The only real solution to security is to admit that bugs happen, and
then mitigate them by having multiple layers.” And our layer is enforced
in kernel space, therefore, there are no existing other attacks that can
be inflected in the user program except for the kernel exploit.


\subsubsection{Scan Base Image}
We scan all the layers which construct the image of the container recursively.
All containers are images in execution, that is we can treat the container
as an image in runtime. Therefore, the layers of image construction have to
be trusted.

For a general image $I_i$ which has been constructed in $n$ layers $L_i$,
$\forall i \le n$, $n \in \mathbb{N}$, we can use the spotbugs\footnote{\url{https://spotbugs.github.io/}}
or the other bug-scanning tools to ensure that the software is a bugless program.
The bugless program $p_i$ is in the layer $L_i$ which construct the $I_i$

\subsubsection{Building and Signing}

We will execute the developer's unit tests and the integration test in the build time.
We catch all the system calls $s_i$ by the BoSC\cite{1495942} method,
and generate the $S = \{s_1, s_2, \dots s_i \dots s_n\}$ set from the
program's $n$ system calls, $S \subseteq \mathbb{S}$, the $\mathbb{S}$ is all the system
calls that the kernel supported.  We wrote a driver to parse the $S$ into
a whitelist filter of seccomp's policy $P$.

% To catch the system calls of the programs in the container, we use a 

Through the workflow above, the $L_i$'s security is almost surely enough.
Then we sign our certificate $C$ and the policy $R$ to the image $I_i$, which
is constructed by those trusted layers $L_i$ into $\hat I_{i}$. That is
$\hat I_{i} = C(P \oplus \Sigma_{\forall i} L_i)$.

\subsubsection{Check Image and Policy}
When we deploy the $\hat{I_i}$ into an active machine, we have to check the $C$ of
$\hat{I_i}$ is valid for signer's trusted verification server.

The verification server can check the certificate $C$'s integrity and encrypt those
checking results by the server's private key $P_{VK}$ to the active machine. The active
machine will also check the certificate $C'$ from the verification server bidirectionally.

And we register our policy $P$ into the active machine's kernel to limit the $\hat{I_i}$
launched by the user in runtime, that is the container.

\subsubsection{Enforce the Policy}

The kernel of the active machine can help us to guarantee the policy $P$ is enforced in
kernel space. Since the container is launched by the user, the policy $P$
has been invoked in each system calls of the container. Because the policy $P$ is a whitelist,
all of the other system calls which do not belong to the signed container's application
would send a permission denied signal from the kernel.

\subsection{Rolling Updates}

The rolling update is a trend of software engineering products, which is
also named agile software development. Eric S. Raymond formulated
the Linus's law in \emph{The Cathedral and the Bazaar}\cite{9780596001087}.
We give enough eyeballs and layers, all bugs or vulnerabilities are shallow in our
healthcare data exchange system. Therefore the container can be secure eventually.
