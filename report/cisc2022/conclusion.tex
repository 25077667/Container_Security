\section{Conclusion}

We can see the comparison results in virtual machine and container are
significantly indifferent order of time-consuming. There is no existence
of the gVisor's result because the gVisor was not able to launch the
IBM/FHIR server system, which is the target of our research.
We also expect the gVisor might run faster significantly than the virtual
machine, however, our target cannot be launched successfully in
gVisor's sandbox.

We thought there might have been some race condition bugs via JWE(JAVA Web
Engine) in gVisor. We found the IBM/FHIR server return an error code $141$
while it launching. However, we did the same configuration in Docker with
our policy and raw gVisor. Therefore, we thought the gVisor did not do
well in supporting all system calls.

And the time complexity of the virtual machine is significantly different from
the container. We propose a hypothesis of the time complexity of the virtual
machine, because there are more page fault events and limited by the
throughput of virtual machine device driver\cite{10.5555/1267569.1267570,7095802}.

The proposed scheme is nearly zero-overhead protection in the kernel, and
the policy is auto-generated and conformed in the build time. It could help
developers to deploy into a secure environment with no pain. This paper also
benchmarked the protection via virtual machine and our proposed scheme in
the container. It can be concluded that it is a good solution if we want
low latency and security.

% \paragraph*{Better Architecture}
% To better profile the IBM/FHIR server, we can start with the Hooked JVM,
% which prevents unexpected objects from being generated at execution time,
% such as fetching unspecified settings from JDBC. To make this method come
% true, we will need to do two things: 1. JVM Hook, 2. Pre-parse the relationship
% between Java bytecode and system calls.

% In order to better provide the maintenance of information security for specific
% applications, containerization is one of the means. In the face of more detailed
% security controls, we should especially focus on more detailed patterns. The
% collection system call is of course more versatile, but this is where we can
% design better.

% We think the problem with delay is the hypervisor's driver buffer bottleneck, but
% there might be still too many variables in the middle. We need more analysis of
% hypervisor and hardware communication mechanisms' formula to determine the root
% causes.

% \paragraph*{Future Machine Learning in Kernel}
% Each FHIR request will conform to a certain format, and even when it is highly
% parallel, it will have a certain pattern. We can start from the FHIR container
% and put recurrent neural network or Hidden Markov Model into the kernel via ebpf.
% Perhaps we will have more accurate and flexible containers to protect the health
% and medical information exchange system.
