\section{Conclusion}

We can see that the comparison results in virtual machine and container are
significantly indifferent order of time-consuming. There is no existence
of the gVisor's result because the gVisor was not able to launch the
IBM/FHIR server system, which is the target of our research.
We also expect that the gVisor might run faster significantly than the virtual
machine; however, our target cannot be launched successfully in
gVisor's sandbox.
We thought that there might have been some race condition bugs via JWE (JAVA Web
Engine) in gVisor, such that the gVisor did not do well in supporting all system calls.

And the time complexity of the virtual machine is significantly different from
the container. We propose a hypothesis of the time complexity of the virtual
machine, because there are more page fault events and the throughput limitation
of virtual machine device driver \cite{10.5555/1267569.1267570,7095802}.

The proposed scheme is nearly zero-overhead protection in the kernel, and
the policy is auto-generated and conformed in the build time. It could help
developers to deploy into a secure environment with no pain. This paper also
benchmarked the protection via virtual machine and our proposed scheme in
the container. It can be concluded that it is a good solution if we want
low latency and security.

