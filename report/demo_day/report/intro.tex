\chapter{Introduction}

\section{Container and Linux Kernel}

The container is a secondary product of the operating system in the past 20 years.
The FreeBSD develops `Jails' in 1999, and the Solaris develops `Zones' in 2004.
Linux also took this idea into the Linux kernel, which is named cgroups (2007),
the capabilities (2003), and seccomp (2005). However, why the Linux breaks this
technology into many parts? This is because they had discussed:
"Why Should a System Administrator Upgrade?" in 2001
\footnote{Version 2.4 of the LINUX KERNEL--Why Should a System Administrator Upgrade?
    \url{https://www.informit.com/articles/article.aspx?p=20667}}.
The Linux kernel almost entered the development path of "upgrade for demand" like
Microsoft Windows, and deviated from the original path of "providing a mechanism
but not a strategy" of the original Linux kernel. \\

While Linux were spreading in various server or distributed system, the
Linux community got more pull requests to solved the scalability and virtualization
issues \cite{267148}. However, they avoided confusion caused by multiple meanings of
the term "container" in the Linux kernel context. In kernel version 2.6.24 (2007)
\footnote{Notes from a container: \url{https://lwn.net/Articles/256389/}},
control groups functionality was merged into the mainline,
which is designed for an administrator (or administrative daemon) to organize processes
into hierarchies of containers; each hierarchy is managed by a subsystem. Moreover, the
cgroups was rewrote into cgroups-v2 in Linux kernel 4.5 (2015)
\footnote{Control Group v2: \url{https://www.kernel.org/doc/Documentation/cgroup-v2.txt}}.\\

The first and most complete implementation of the Linux container manager was LXC
(Linux Containers). It was implemented in 2008 using cgroups and namespaces,
and it runs on a single Linux kernel without requiring any patches. LXC provides
a new view and imagination of virtualized services without any hypervisor. In 2016,
Docker replaced LXC with "libcontainer", which was written in the Go programming language.
Docker combined features in a new, more attractive way and made Linux containers popular.\\

The secondary product of the operating system, containers, offering many advantages:
they enable you to "build once, run anywhere." Docker does this by bundling
applications with all their dependencies into one package and isolating applications
from the rest of the machine on which they're running. Therefore, this research
is based on docker container to propose a scheme of healthcare data exchange system's
security.

\section{FHIR}
FHIR is a standard for healthcare data exchange. The FHIR standard will be used in
Taiwan in the near future. FHIR will be used to provide PHR (Personal Healthcare Records)
in Taiwan. Therefore, we choose the most popular standard "FHIR" for the target of
the healthcare data exchange system.

\subsection{RESTful API and Data Structure}
REST (Representational State Transfer) is a stateless reliable web API, which is based
on HTTP methods to access resources or data via URL parameters and the use of JSON or
XML format to transmit queries. Because the RESTful is stateless, the client should
keep their information (i.e. cookies) by themself.\\


FHIR has features: RESTful and data structure, make our research and benchmarks more accurate
and reliable.
Statelessness is a developer-friendly feature, the developer and the tester would not
to design a complex state machine on the server-side or generating test files. And the FHIR
takes RESTful as standard. Moreover, FHIR standard declared the `StructureDefinition'
\footnote{FHIR Resource Structure Definition: \url{http://www.hl7.org/fhir/structuredefinition.html}}.
These structure definitions are used to describe both the content defined in the FHIR
specification itself - Resources, data types, the underlying infrastructural types, and
also are used to describe how these structures are used in implementations.

\subsection{Why IBM FHIR server}
There are many applications using IBM's FHIR server as the base component of the EHR
(Electronic Health Records) system to communicate with the other various databases.
Take it for example that the NextCloud's EHR service, Taipei Veterans General Hospital,
and AWS Cloud are using the FHIR server in a container for subroutine service.\\

NextCloud is an open-source and self-hosted productivity platform for users.
Many people caring about their privacy issues distrust the FAAMG (Facebook,
Amazon, Apple, Microsoft, Google), so they are using NextCloud to keep their privacy
on their own. Therefore, they are eager to have a secure EHR system for their
PHR
\footnote{\href{https://www.youtube.com/watch?v=AAP4N3KyLmM}{Richard Stallman talks about IoT}}.

\section{Data and Privacy}
\TODO{Section}
