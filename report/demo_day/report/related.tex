\chapter{Related Work}

\section{Collecting System Calls}
There are several pieces of research to detect intrusions or unexpected behaviors
by collecting the system calls methods in runtime \cite{inproceedings,9307722,7809699,7796855}.
Amr S. Abed et al.\cite{inproceedings} proposed a real-time host-based intrusion
detection system in a container, which is based on system call monitoring. They use
the `strace' command to collect a behavior log to a system call parser. Then use the
BoSC (Bag of System Calls) \cite{1495942} to classify is it a normal behavior in
the database.

The BoSC technique is a frequency-based detection tip. Kang et al.\cite{1495942} defined
those distinct system calls in $\{c_1, c_2, \dots, c_n \}$, For all system call $s_i$
had been called in $c_i$ times. And they use Na\"ive Bayes classification to deduce if
it is unexpected behavior. Then the Amr S. Abed et al. give the false positive rate
around 2\% in $O(S+n_k)$ epochs to the MySQL database \cite{inproceedings}.
\begin{itemize}
    \item Epoch Size ($S$): The total number of system calls in one epoch.
    \item $n_k$: It is the size of the database after epoch $k$.
\end{itemize}
However, the BoSC is running in user space, even though it is a background service running
on the same host kernel. It might have heavy constant time costs of copying data from
user to kernel and kernel to user by the `copy\_to\_user()' and `copy\_from\_user()' calls.\\

Mohamed Azab et al.\cite{7809699,7796855} takes a mathematical model to simulate the
smart moving target defense for Linux container resiliency. Considering an `ESCAPE' model
is the interaction between attackers and their target containers as a “predator searching
for a prey” search game. This search game has 3 modules: behavior monitoring, the
checkpoint/restore, and the live migration modules.
This model is running on the same host and the same attacking surface because they considered
the containers (prey) are running on the same machine with some migration probability.

They show the survival rate in Amr S. Abed et al.\cite{inproceedings} model for some
zero-day vulnerabilities in different types and numbers machines. Mohamed Azab et al.
\cite{7809699,7796855} concluded that an IDS could detect and avoid mobile continually-growing
attacks efficiently by the `ESCAPE' model with collecting system calls.\\

\section{Fine-grained Permission Control}
\subsection{Capabilities}

There are 49 different capabilities in today's Linux kernel 5.13
\footnote{\url{https://man7.org/linux/man-pages/man7/capabilities.7.html}}. A capability can be
assigned to a task (i.e thread or process) to determine if the task can use the fine-granted
system calls. For example, we give a thread CAP\_SYS\_BOOT, then the thread can use the
reboot \footnote{\url{https://man7.org/linux/man-pages/man2/reboot.2.html}} and the
kexec\_load \footnote{\url{https://man7.org/linux/man-pages/man2/kexec\_load.2.html}} system call.\\

Xin Lin et al.\cite{10.1145/3274694.3274720} collected 27 CVE vulnerabilities that could cause
the privilege escalation attacks. There are only three vulnerabilities that could bypass the
capabilities protection of the Linux kernel. And the other 24 escalation vulnerabilities, could
be filtered by the fine-grained permission control with capabilities. Those three (CVE-2016-8655,
CVE-2017-5123, CVE-2017-7308) bypassed capabilities vulnerabilities are attacking kernel-level
race conditions.

The CVE-2016-8655\footnote{\url{https://cve.mitre.org/cgi-bin/cvename.cgi?name=CVE-2016-8655}}
is a bug in net/packet/af\_packet.c. We often use the CAP\_NET\_RAW namespace in the container
to make unprivileged users be able to use some privileged net-util commands. The bug is that
there exists a race condition probability to race the unauthorized data inside packet\_set\_ring()
and packet\_setsockopt()\footnote{Linux af\_packet.c race condition (local root) \url{https://lwn.net/Articles/708319/}}
such that  there is a chance to modify the socket version to TPACKET\_$V_1$ before the
packet\_set\_ring function. However, it would be ‘kfree‘ the timer in the TPACKET\_$V_1$.
Then we can take the timer, which  is used after free, to control the SLAB adopter to write
the st\_uid by itself \footnote{CVE-2016-8655 PoC: \url{https://www.exploit-db.com/exploits/40871}}.

\subsection{Linux Secure Module}
There were many pieces of research designing great rules of the LSM (Linux secure module). LSM is a
mandatory access control framework in kernel, which is supported from Linux 2.6 (2004). The
AppArmor\footnote{\url{https://apparmor.net/}} and SELinux\footnote{\url{https://github.com/SELinuxProject/selinux}}
are common LSM in the kernel. All the
Android devices are fully enforced the SELinux after Android 5.0.
\footnote{\url{https://source.android.com/security/selinux\#supporting\_documentation}}

The SIDs (Security IDs) and permissions, which are the identifiers for access control policies,
make up the security policy of SELinux or AppArmor. Files and directories, which are the actual protected
objects for the SID, are mapped to the SELinux or AppArmor of each system \cite{Smalley2003ImplementingSA,x11-SELinux,quteprints30563}.
The SELinux is much more incredibly complex than AppArmor, but with this complexity, you have more
control over how tasks are isolated. However, the AppArmor is so straightforward that people can
write the configuration profile by themself. \\

SUNG-HWA HAN et al.\cite{9184912} had proposed an architecture to enforce the access
control of image's layers. Because the docker engine does not guarantee the layers could
not be modified by the host environment. Therefore, if we give a container privileged
permission, it could modify the layers of images. The research \cite{9184912} is using
the LSM's policy table to enforce the access control of the file system in the kernel.

This paper \cite{9184912} shows that the performance is almost the same as raw SELinux,
and the branch costs of indexing the policy table could be a constant value in the
CPU rate measurement. However, the research is only be measured in overlay2 file system.
SUNG-HWA HAN et al. said if it has the same performance in AUFS or the other file systems,
the above results could be more reliable. \\

Yuqiong Sun et al.\cite{217614} proposed separate the security namespace. Each container
can route their operation to different security namespaces for their "comment". Each
involved in the security namespace independently makes a security decision, and the
operation is allowed only if all the secure namespaces allow.


% TODO: Security Namespace : Making Linux Security Frameworks Available to Containers
The policy engine and the security namespace are implemented in \dots


\section{Recently Exploited Vulnerabilities}
In this section, we will mention and review some `High' or `Critical' vulnerabilities
in CVSS (Common Vulnerability Scoring System).

% TODO:
TBD\dots

\subsection{Five Stages of Malware}
We had been inspired by the quark engine\footnote{\url{https://quark-engine.readthedocs.io/en/latest/}},
which is an open-source malware scoring system for Android APK files. The quark
engine had been developed from the Taiwan Criminal Law's five stages:
(\Rn{1}) determination, (\Rn{2}) conspiracy, (\Rn{3}) preparation, (\Rn{4}) start, (\Rn{5}) practice.\\

We also can use these five stages to give the malware limitation to exploit the vulnerabilities.

% TODO:
TBD\dots

\subsection{Case studies}

% TODO: 
\subsubsection{The Dirty CoW}
Delwar Alam, et al.\cite{8019988} showed the race condition and the mechanism of "Copy on
Write". "Copy on Write" is a resource-management technique used in computer programming
to efficiently implement a "duplicate" or "copy" operation on modifiable resources\cite{6394426}.
It is often inspired when ‘fork‘ or ‘mmap‘.

\paragraph{Mechanism}
TBD\dots

\subsubsection{CVE-2016-8655 series}
TBD\dots

\section{Virtual Environment Performance Benchmark}
There is a trend of applications are developed or deployed into microservice in a virtual
environment since 2008. And the performance benchmark of applications in the virtual
environment becomes more and more critical.

Therefore, there are many pieces of research shows how to evaluate the performance when
using containers or the other virtual infrastructures\cite{7371699,KOZHIRBAYEV2017175,7095802,234857}.
They are comparing the throughput, latency, and QoS for memory IO, or cryptography 
algorithms calculating costs.\\

Ethan G. Young, et al.\cite{234857} showed the gVisor costs: $2.2\times$ system call overhead,
$2.5\times$ memory allocation latency, and $216\times$ \textbf{slower} than raw system on
complex file opening.

% TODO: 
TBD\dots
